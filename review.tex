\documentclass{article}
\usepackage[utf8]{inputenc}
\usepackage{amsmath}
\usepackage{graphicx}
\usepackage{{booktabs}}
\usepackage{{longtable}}
\usepackage{indentfirst}

\usepackage{subfiles}

\input{glyphtounicode}
\pdfgentounicode=1 % make ATS friendly


\usepackage[margin=1in]{geometry}

\title{CSE6644 - Iterative Methods - Practice Exam Questions}
\author{Eric Fowler \\}
\date{\today}

\begin{document}

\maketitle

\section{Topics to Study}

\subsection{GMRES (restarted/non-restarted)}

\subsection{Arnoldi}

\subsection{Lanczos}

\subsection{Multigrid (V-cycle)}

\subsection{Broyden's Algorithm}

\subsection{BFGS}

\subsection{Levinberg-Marquardt}

\subsection{Fixed-Point Iteration}

\subsection{Gauss-Newton}

\begin{enumerate}

\section{Questions}




% \subfile{qEric}

\item Prove that Arnolid is a residual projection method.

\item Prove that Lanczos is a residual projection method

\item How do you get $e_1$ in the GMRES method?

\item If CG is based on Lanczos, which is based on Arnoldi, and Arnoldi's method is residual projection, then why is CG error projection?

\item Probing

\item The function $||\nabla f(x_+ ) || < \epsilon$ can be used as a stopping condition, but it can 
Why would $\frac{\nabla f(x_i)x_i}{f(x)}$ be a better choice?

\item $f(x) = sin(x)$ If $\epsilon$ is a small positive number and $x_0 = -\epsilon$, show that Newton's method gives $x_1 \cong -\frac{1}{\epsilon}$. (Dennis and Schnabel 5.16.

\end{enumerate}

\end{document}